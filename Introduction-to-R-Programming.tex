% Options for packages loaded elsewhere
\PassOptionsToPackage{unicode}{hyperref}
\PassOptionsToPackage{hyphens}{url}
%
\documentclass[
]{article}
\usepackage{amsmath,amssymb}
\usepackage{lmodern}
\usepackage{iftex}
\ifPDFTeX
  \usepackage[T1]{fontenc}
  \usepackage[utf8]{inputenc}
  \usepackage{textcomp} % provide euro and other symbols
\else % if luatex or xetex
  \usepackage{unicode-math}
  \defaultfontfeatures{Scale=MatchLowercase}
  \defaultfontfeatures[\rmfamily]{Ligatures=TeX,Scale=1}
\fi
% Use upquote if available, for straight quotes in verbatim environments
\IfFileExists{upquote.sty}{\usepackage{upquote}}{}
\IfFileExists{microtype.sty}{% use microtype if available
  \usepackage[]{microtype}
  \UseMicrotypeSet[protrusion]{basicmath} % disable protrusion for tt fonts
}{}
\makeatletter
\@ifundefined{KOMAClassName}{% if non-KOMA class
  \IfFileExists{parskip.sty}{%
    \usepackage{parskip}
  }{% else
    \setlength{\parindent}{0pt}
    \setlength{\parskip}{6pt plus 2pt minus 1pt}}
}{% if KOMA class
  \KOMAoptions{parskip=half}}
\makeatother
\usepackage{xcolor}
\IfFileExists{xurl.sty}{\usepackage{xurl}}{} % add URL line breaks if available
\IfFileExists{bookmark.sty}{\usepackage{bookmark}}{\usepackage{hyperref}}
\hypersetup{
  hidelinks,
  pdfcreator={LaTeX via pandoc}}
\urlstyle{same} % disable monospaced font for URLs
\usepackage[margin=1in]{geometry}
\usepackage{color}
\usepackage{fancyvrb}
\newcommand{\VerbBar}{|}
\newcommand{\VERB}{\Verb[commandchars=\\\{\}]}
\DefineVerbatimEnvironment{Highlighting}{Verbatim}{commandchars=\\\{\}}
% Add ',fontsize=\small' for more characters per line
\usepackage{framed}
\definecolor{shadecolor}{RGB}{248,248,248}
\newenvironment{Shaded}{\begin{snugshade}}{\end{snugshade}}
\newcommand{\AlertTok}[1]{\textcolor[rgb]{0.94,0.16,0.16}{#1}}
\newcommand{\AnnotationTok}[1]{\textcolor[rgb]{0.56,0.35,0.01}{\textbf{\textit{#1}}}}
\newcommand{\AttributeTok}[1]{\textcolor[rgb]{0.77,0.63,0.00}{#1}}
\newcommand{\BaseNTok}[1]{\textcolor[rgb]{0.00,0.00,0.81}{#1}}
\newcommand{\BuiltInTok}[1]{#1}
\newcommand{\CharTok}[1]{\textcolor[rgb]{0.31,0.60,0.02}{#1}}
\newcommand{\CommentTok}[1]{\textcolor[rgb]{0.56,0.35,0.01}{\textit{#1}}}
\newcommand{\CommentVarTok}[1]{\textcolor[rgb]{0.56,0.35,0.01}{\textbf{\textit{#1}}}}
\newcommand{\ConstantTok}[1]{\textcolor[rgb]{0.00,0.00,0.00}{#1}}
\newcommand{\ControlFlowTok}[1]{\textcolor[rgb]{0.13,0.29,0.53}{\textbf{#1}}}
\newcommand{\DataTypeTok}[1]{\textcolor[rgb]{0.13,0.29,0.53}{#1}}
\newcommand{\DecValTok}[1]{\textcolor[rgb]{0.00,0.00,0.81}{#1}}
\newcommand{\DocumentationTok}[1]{\textcolor[rgb]{0.56,0.35,0.01}{\textbf{\textit{#1}}}}
\newcommand{\ErrorTok}[1]{\textcolor[rgb]{0.64,0.00,0.00}{\textbf{#1}}}
\newcommand{\ExtensionTok}[1]{#1}
\newcommand{\FloatTok}[1]{\textcolor[rgb]{0.00,0.00,0.81}{#1}}
\newcommand{\FunctionTok}[1]{\textcolor[rgb]{0.00,0.00,0.00}{#1}}
\newcommand{\ImportTok}[1]{#1}
\newcommand{\InformationTok}[1]{\textcolor[rgb]{0.56,0.35,0.01}{\textbf{\textit{#1}}}}
\newcommand{\KeywordTok}[1]{\textcolor[rgb]{0.13,0.29,0.53}{\textbf{#1}}}
\newcommand{\NormalTok}[1]{#1}
\newcommand{\OperatorTok}[1]{\textcolor[rgb]{0.81,0.36,0.00}{\textbf{#1}}}
\newcommand{\OtherTok}[1]{\textcolor[rgb]{0.56,0.35,0.01}{#1}}
\newcommand{\PreprocessorTok}[1]{\textcolor[rgb]{0.56,0.35,0.01}{\textit{#1}}}
\newcommand{\RegionMarkerTok}[1]{#1}
\newcommand{\SpecialCharTok}[1]{\textcolor[rgb]{0.00,0.00,0.00}{#1}}
\newcommand{\SpecialStringTok}[1]{\textcolor[rgb]{0.31,0.60,0.02}{#1}}
\newcommand{\StringTok}[1]{\textcolor[rgb]{0.31,0.60,0.02}{#1}}
\newcommand{\VariableTok}[1]{\textcolor[rgb]{0.00,0.00,0.00}{#1}}
\newcommand{\VerbatimStringTok}[1]{\textcolor[rgb]{0.31,0.60,0.02}{#1}}
\newcommand{\WarningTok}[1]{\textcolor[rgb]{0.56,0.35,0.01}{\textbf{\textit{#1}}}}
\usepackage{graphicx}
\makeatletter
\def\maxwidth{\ifdim\Gin@nat@width>\linewidth\linewidth\else\Gin@nat@width\fi}
\def\maxheight{\ifdim\Gin@nat@height>\textheight\textheight\else\Gin@nat@height\fi}
\makeatother
% Scale images if necessary, so that they will not overflow the page
% margins by default, and it is still possible to overwrite the defaults
% using explicit options in \includegraphics[width, height, ...]{}
\setkeys{Gin}{width=\maxwidth,height=\maxheight,keepaspectratio}
% Set default figure placement to htbp
\makeatletter
\def\fps@figure{htbp}
\makeatother
\setlength{\emergencystretch}{3em} % prevent overfull lines
\providecommand{\tightlist}{%
  \setlength{\itemsep}{0pt}\setlength{\parskip}{0pt}}
\setcounter{secnumdepth}{-\maxdimen} % remove section numbering
\ifLuaTeX
  \usepackage{selnolig}  % disable illegal ligatures
\fi

\author{}
\date{\vspace{-2.5em}}

\begin{document}

\hypertarget{problem-statement}{%
\section{Problem Statement}\label{problem-statement}}

ShippingtoMali, a shipping company, provides shipping to Mali for items
bought on Amazon at a rate of 4 dollars for the first item and \$2 for
each subsequent item with each of those items is also subject to import
10\% tax. During this time, they also have a \$10 discount for orders
worth above \$500. Write an R program that prints out how much one would
spend for both purchase and shipping of the given items (`item 1', `item
2', `item 3', `item 4', `item 5', `item 6', `item 7') from the given
data that is stored in the vectors or lists. We can write two functions
that will help provide a solution to this problem.

● One function will iterate over the price list/vector (120, 32, 99, 49,
99, 75, 30) applying the respective tax and appending the result to the
total\_cost list/vector.

● The other function will take the number of items in an order as its
only parameter then return the cost of shipping the function's result.
This function would only display the shipping charge

\begin{Shaded}
\begin{Highlighting}[]
\CommentTok{\# Let\textquotesingle{}s create a function with parameters}
\CommentTok{\# {-}{-}{-}}
\CommentTok{\#}

\CommentTok{\#Returns the prices of the items after applying the 10\% tax}
\NormalTok{after\_tax\_prices }\OtherTok{\textless{}{-}} \ControlFlowTok{function}\NormalTok{()\{}
\NormalTok{  total\_prices }\OtherTok{\textless{}{-}} \FunctionTok{c}\NormalTok{(}\AttributeTok{length =} \DecValTok{7}\NormalTok{)}
\NormalTok{  prices }\OtherTok{\textless{}{-}} \FunctionTok{c}\NormalTok{(}\DecValTok{120}\NormalTok{, }\DecValTok{32}\NormalTok{, }\DecValTok{99}\NormalTok{, }\DecValTok{49}\NormalTok{, }\DecValTok{99}\NormalTok{, }\DecValTok{75}\NormalTok{, }\DecValTok{30}\NormalTok{)}
\NormalTok{  i }\OtherTok{=} \DecValTok{1}
  \ControlFlowTok{while}\NormalTok{(i }\SpecialCharTok{\textless{}=} \FunctionTok{length}\NormalTok{(prices)) \{}
\NormalTok{    total\_prices[[i]] }\OtherTok{\textless{}{-}}\NormalTok{ prices[i]}\SpecialCharTok{*}\FloatTok{1.1}
\NormalTok{    i }\OtherTok{\textless{}{-}}\NormalTok{ i }\SpecialCharTok{+} \DecValTok{1}
\NormalTok{  \}}
  \FunctionTok{return}\NormalTok{(total\_prices)}
\NormalTok{\}}
\FunctionTok{print}\NormalTok{(}\StringTok{"Prices after taxation"}\NormalTok{)}
\end{Highlighting}
\end{Shaded}

\begin{verbatim}
## [1] "Prices after taxation"
\end{verbatim}

\begin{Shaded}
\begin{Highlighting}[]
\FunctionTok{after\_tax\_prices}\NormalTok{()}
\end{Highlighting}
\end{Shaded}

\begin{verbatim}
## length                                           
##  132.0   35.2  108.9   53.9  108.9   82.5   33.0
\end{verbatim}

\begin{Shaded}
\begin{Highlighting}[]
\NormalTok{cost\_of\_shipping }\OtherTok{\textless{}{-}} \ControlFlowTok{function}\NormalTok{(number\_of\_items)\{}
\NormalTok{  shipping\_cost }\OtherTok{\textless{}{-}} \DecValTok{0}
  \ControlFlowTok{if}\NormalTok{ (number\_of\_items }\SpecialCharTok{==} \DecValTok{1}\NormalTok{) \{}
\NormalTok{    shipping\_cost }\OtherTok{\textless{}{-}} \DecValTok{4}
\NormalTok{  \} }\ControlFlowTok{else}\NormalTok{ \{}
\NormalTok{    shipping\_cost }\OtherTok{\textless{}{-}} \DecValTok{4} \SpecialCharTok{+}\NormalTok{ ((number\_of\_items}\DecValTok{{-}1}\NormalTok{)}\SpecialCharTok{*}\DecValTok{2}\NormalTok{)}
\NormalTok{  \}}
  
  \FunctionTok{return}\NormalTok{(shipping\_cost)}
\NormalTok{\}}
\FunctionTok{print}\NormalTok{(}\StringTok{"cost of shipping"}\NormalTok{)}
\end{Highlighting}
\end{Shaded}

\begin{verbatim}
## [1] "cost of shipping"
\end{verbatim}

\begin{Shaded}
\begin{Highlighting}[]
\FunctionTok{cost\_of\_shipping}\NormalTok{(}\DecValTok{3}\NormalTok{)}
\end{Highlighting}
\end{Shaded}

\begin{verbatim}
## [1] 8
\end{verbatim}

\begin{Shaded}
\begin{Highlighting}[]
\NormalTok{total\_cost\_of\_shipping }\OtherTok{\textless{}{-}}\ControlFlowTok{function}\NormalTok{(number\_of\_items) \{}
\NormalTok{  total\_cost\_of\_shipping }\OtherTok{\textless{}{-}} \FunctionTok{cost\_of\_shipping}\NormalTok{(number\_of\_items) }\SpecialCharTok{+} \FunctionTok{sum}\NormalTok{(}\FunctionTok{after\_tax\_prices}\NormalTok{()[}\DecValTok{1}\SpecialCharTok{:}\NormalTok{number\_of\_items])}
  
  \CommentTok{\#This checks if the customer qualifies for the discount}
  \ControlFlowTok{if}\NormalTok{(total\_cost\_of\_shipping }\SpecialCharTok{\textgreater{}=} \DecValTok{500}\NormalTok{) \{}
\NormalTok{    total\_cost\_of\_shipping }\OtherTok{\textless{}{-}}\NormalTok{ total\_cost\_of\_shipping }\SpecialCharTok{{-}} \DecValTok{10}
\NormalTok{  \}}
  \FunctionTok{return}\NormalTok{(total\_cost\_of\_shipping)}
\NormalTok{\}}

\FunctionTok{print}\NormalTok{(}\StringTok{"cost of shipping plus prices (total cost)"}\NormalTok{)}
\end{Highlighting}
\end{Shaded}

\begin{verbatim}
## [1] "cost of shipping plus prices (total cost)"
\end{verbatim}

\begin{Shaded}
\begin{Highlighting}[]
\FunctionTok{total\_cost\_of\_shipping}\NormalTok{(}\DecValTok{3}\NormalTok{)}
\end{Highlighting}
\end{Shaded}

\begin{verbatim}
## [1] 284.1
\end{verbatim}

\end{document}
